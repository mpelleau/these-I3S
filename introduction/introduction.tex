% !TEX root = ../sommaire.tex

\chapter{Introduction}

Cette classe \LaTeX est basée sur la classe these-LINA écrite par Frédéric Goualard et modifiée par Alban Mancheron. La classe a été modifiée pour correspondre au modèle de thèse demandé par l'ED STIC de l'Université Côte d'Azur.

Pour les autres ED, il suffit de supprimer le fichier \texttt{visuel.jpg} et de renommer le visuel correspondant à l'ED en \texttt{visuel.jpg}.

\section{Citations}

Dans ce document on a créé 3 bibliographie :
\begin{itemize}
  \item la bibliographie pour les publications personnelles,
  \item la bibliographie pour les pages web,
  \item et la bibliographie générale.
\end{itemize}

Elles sont définies au début du fichier \texttt{sommaire.tex}.

\begin{framed}
\begin{verbatim}
% Biblio pour les pages webs
\newcites{web}{Pages web}
% Biblio pour mes publications
\newcites{mine}{Mes publications}
\end{verbatim}\vspace{-0.5em}
\end{framed}

On utilise pour cela le package \texttt{multibib}, vous pouvez ajouter les votre ou renommer celles existantes, il ne faut pas oublier de modifier le fichier \texttt{Makefile} pour compiler les nouvelles bibliographies ou supprimer la compilation de celles que vous n'utiliser pas.

\paragraph{Les publications dans la bibliographie générale} doivent être citées avec la macro \texttt{\textbackslash cite\{key\}}.

\begin{framed}
\noindent\cite{T2012}\vspace{-0.5em}
\begin{verbatim}\cite{T2012}\end{verbatim}\vspace{-0.75em}
\end{framed}

\paragraph{Les publications pour la section pages web} doivent être citées avec la macro \texttt{\textbackslash citeweb\{key\}}.

\begin{framed}
\noindent\citeweb{AC9} \vspace{-0.5em}
\begin{verbatim}\citeweb{AC9}\end{verbatim}\vspace{-0.75em}
\end{framed}

\paragraph{Les publications pour la section correspondant à vos publications} doivent être citées avec la macro \texttt{\textbackslash citemine\{key\}}.

\begin{framed}
\noindent\citemine{DHHS2011} \vspace{-0.5em}
\begin{verbatim}\citemine{DHHS2011}\end{verbatim}\vspace{-0.75em}
\end{framed}

Pour référencer vos publications par la suite, vous pouvez utiliser la macro \texttt{\textbackslash cite} ou la macro \texttt{\textbackslash citemine}.
Si vous souhaitez que votre publication apparaisse dans la bibliographie générale à la fin du manuscript vous pouvez soit la citer avec \texttt{\textbackslash cite} ou l'insérer avec \texttt{\textbackslash nocite}. \nocite{DHHS2011} 

Attention si vous utilisez un style de bibliographie numérotant les références, vous aurez des problèmes de numérotation si une publication est à la fois citée en utilisant la macro \texttt{\textbackslash cite} et la macro \texttt{\textbackslash citemine}.


Chaque bibliographie peut ensuite être affichée en utilisant les lignes suivantes :

\begin{framed}
\begin{verbatim}\bibliographystyle<s>{apalike}
\bibliography<s>{biblio}
\end{verbatim}\vspace{-0.5em}
\end{framed}

où \verb|<s>| correspond à la bibliographie que l'on veut afficher.


\section{Présentation de la problématique}

\section{Organisation de ce manuscrit}

\section{Nos contributions}

Nous avons fait ça qui a fait l'objet de la publication \citemine{DHHS2011}. Puis nous avons fait ci qui a fait l'objet de la publication \citemine{HHSD2011}. Finalement nous avons fait cela qui a fait l'objet de la publication \citemine{HHSD2012}.
 

  
\renewcommand{\bibtitle}{\section*{\refname}}
\bibliographystylemine{apalike}
\bibliographymine{biblio}


