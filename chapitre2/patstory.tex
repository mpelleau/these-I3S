% !TEX root = ../sommaire.tex

\section{La pâtisserie de la préhistoire à aujourd'hui}

Comme chacun sait la pâtisserie a toujours existé, même les hommes et femmes préhistoriques aimaient finir leurs repas sur une note sucrée.

\subsection{Les premiers gâteaux}


\subsubsection{La première tarte}

Rappelons d'abord la définition de la tarte telle que donnée dans \cite{S2012}.

\begin{definition}[Tarte]
  Une tarte, c'est comme un donuts mais sans trous et avec des fois des fruits dessus.
\end{definition}

\begin{example}[Tarte]
  La tarte aux citrons et une tarte.
\end{example}

\begin{remark}
  Notons que la tarte au chocolat est aussi une tarte par contre le mille-feuilles n'est pas une tarte.
\end{remark}

Après de très nombreux travaux on peut aujourd'hui affirmer que la première tarte était une tarte aux citrons, et vous devez nous croire parce qu'on porte des lunettes de soleil \cite{C2012}.

\subsubsection{Le premier éclair}

\subsubsection{Le premier flan}

\subsubsection{La première religieuse}


\subsection{Évolution de la pâtisserie au cours du temps}

\subsection{L'arrivée du Macaron}

\subsection{Conclusion}